%%-----------------------------------------------------------------------
% O argumento chapter=TITLE define que todos os títulos de capítulos
% saiam em caixa alta (maiúsculas). Se não quiser caixa alta, basta
% remover este argumento.
%\documentclass[12pt,oneside,a4paper,chapter=TITLE,
			  %english,brazil]{abntex2}

%\documentclass[12pt,oneside,a4paper,chapter=TITLE]{abntex2}


%\documentclass[12pt,oneside,a4paper,chapter=TITLE]{abntex2}
%%-----------------------------------------------------------------------
% Carregando a padronização para a capa, contra-capa e folha de
% assinaturas adotada pela UEPB
\usepackage{TCC_UEPB}
\usepackage{graphicx}
\usepackage{sectsty}
%\pagenumbering{arabic}
%%-----------------------------------------------------------------------
% Pacotes básicos 
\usepackage{mathptmx}		% Usa a fonte Times New Roman
\usepackage[T1]{fontenc}	% Selecao de codigos de fonte.
\usepackage[utf8]{inputenc}	% Cod. do doc. (conversão autom. dos acentos)
\usepackage{lastpage}		% Usado pela Ficha catalográfica
\usepackage{indentfirst}	% Indenta o primeiro parágrafo de cada seção.
\usepackage{float}
\usepackage{forest}
\usepackage{color}			% Controle das cores
\usepackage{graphicx}		% Inclusão de gráficos
\usepackage{microtype} 		% para melhorias de justificação
\usepackage{pdfpages}		% para a inclusão de documentos em PDF.
\usepackage{multirow}       % VARIAS LINHAS NA TABELA
\usepackage{colortbl}       % CORES NA TABELA
\usepackage{hyperref}
\usepackage{titlesec}
%%------------------------------------------------------------------------
% Pacotes adicionais, para definir ambientes para definição, teorema 
% e axioma. Outros ambientes podem ser definidos da mesma forma...
\usepackage{amssymb}     % qed
\usepackage{amsthm}      % Teoremas
\usepackage{amsmath}     % Para o ambiente align (alinhar equações)
\usepackage{thmtools}    % Front end para amsthm (\declaretheorem)
\usepackage{parskip}
\usepackage{titlesec}
\usepackage{hyperref}

%%-----------------------------------------------------------------------
% Definição de ambientes para definição, teorema, etc...
\declaretheorem[style=definition,name=Definição,qed=\textemdash]{definicao}
\declaretheorem[style=plain,name=Teorema,qed=\textnormal{\textemdash}]{teorema}
\declaretheorem[style=plain,name=Axioma,qed=\textnormal{\textemdash}]{axioma}
\declaretheorem[style=plain,name=Lema,qed=\textnormal{\textemdash}]{lema}
% Configuração para seções e subseções
%\titleformat{\section}{\normalfont\Large\bfseries}{\thesection}{1em}{}
\titleformat{\subsection}{\normalfont\large\bfseries}{\thesubsection}{1em}{}

% configurando seções 



%%------------------------------------------------------------------------
% Pacotes de citações
\usepackage[brazilian,hyperpageref]{backref} % Página citada na bibliog.
\usepackage[alf]{abntex2cite}				 % Citações padrão ABNT

%\titlespacing{\chapter}{0pt}{*0}{*0}
%\titlespacing{\section}{0pt}{*0}{*0}
%%-----------------------------------------------------------------------
%% CONFIGURAÇÕES DE PACOTES
%%-----------------------------------------------------------------------

\usepackage{fancyhdr}
\usepackage[bottom=2cm,top=2cm]{geometry} % Ajusta as margens

\pagestyle{plain} % Configura o estilo de página para plain

\fancypagestyle{plain}{%
  \fancyhf{} % Limpa todos os cabeçalhos e rodapés
  \renewcommand{\headrulewidth}{0pt} % Remove a linha do cabeçalho
  \fancyhead[R]{\raisebox{2cm}{\thepage}} % Ajusta a posição vertical (aumentado para -2cm)
}
%%-----------------------------------------------------------------------
% Configurações do pacote backref Usado sem a opção hyperpageref
% de backref
\renewcommand{\backrefpagesname}{Citado na(s) página(s):~}
% Texto padrão antes do número das páginas
\renewcommand{\backref}{}   % Define os textos da citação
\renewcommand*{\backrefalt}[4]{
	\ifcase #1 %
	Nenhuma citação no texto.%
	\or
	Citado na página #2.%
	\else
	Citado #1 vezes nas páginas #2.%
	\fi}%


%%----------------------------------------------------------------------
% Configurações de aparência do PDF final
\makeatletter
\hypersetup{     % informações do PDF
	pdftitle={\@title}, 
	pdfauthor={\@author},
	pdfsubject={\imprimirpreambulo},
	colorlinks=true,   % false: box links; true: color links
	linkcolor=black,   % color of internal links
	citecolor=black,   % color of links to bibliography
	filecolor=black,   % color of file links
	urlcolor=black,    % color of url links
	bookmarksdepth=4
}
\makeatother


%%------------------------------------------------------------------
% Cria uma nova série (de símbolos) para footnotes
\newcounter{savefootnote}
\newcounter{symfootnote}
\newcommand{\symfootnote}[1]{%
	\setcounter{savefootnote}{\value{footnote}}%
	\setcounter{footnote}{\value{symfootnote}}%
	\ifnum\value{footnote}>8\setcounter{footnote}{0}\fi%
	\let\oldthefootnote=\thefootnote%
	\renewcommand{\thefootnote}{\fnsymbol{footnote}}%
	\footnote{#1}%
	\let\thefootnote=\oldthefootnote%
	\setcounter{symfootnote}{\value{footnote}}%
	\setcounter{footnote}{\value{savefootnote}}%
}


% remove informações do  top das paginas 


%\pagestyle{empty} 

%%-----------------------------------------------------------------------
% Informações de dados para CAPA, FOLHA DE ROSTO e FOLHA DE APROVAÇÃO
\titulo{ PREDIÇÃO DA SÍNDROME RESPIRATÓRIA AGUDA GRAVE POR MEIO DE MULTICLASSIFICAÇÃO COM ALGORITMOS DE MACHINE LEARNING}   % Subtítulo apenas se houver
\tituloestrangeiro{PREDICTION OF SEVERE ACUTE RESPIRATORY SYNDROME THROUGH MULTICLASSIFICATION USING MACHINE LEARNING ALGORITHMS} % Title (English - opcional)
\autor{JOSEFERSON DA SILVA BARRETO}   % Digite seu nome aqui
\local{Campina Grande - PB}
\data{2023}                              % Na data coloque apenas o ano
\orientador[Prof. Dr.]{%Ricardo Alves de Olinda & \\
            Tiago Almeida de Oliveira} % Nome do orientador aqui
%\coorientador[]{} % Coorientador (se tiver)

%%-----------------------------------------------------------------------
% Informações sobre campus, centro, depto e curso
% \campus e \curso são obrigatórios e precisam ser preenchidos!!
% Todos os demais são opcionais (para o LaTeX), pois note que o 
% \preambulo é exigido pela Biblioteca na entrega do trabalho.
\campus{Campus I - CAMPINA GRANDE}
\centro{Centro de Ciências e Tecnologia}
\depto{Departamento de Estatística}
\curso{CURSO DE GRADUAÇÃO EM BACHARELADO EM ESTATÍSTICA}
\tipotrabalho{Trabalho de Conclusão de Curso}
\preambulo{
  Trabalho de Conclusão de Curso apresentado ao curso de Bacharelado em Estatística do Departamento de Estatística do Centro de Ciências e Tecnologia da Universidade Estadual da Paraíba, como requisito parcial à obtenção do título de bacharel em Estatística.
}


%%-----------------------------------------------------------------------
% compila o indice
\makeindex

\usepackage{pdfpages}

%\pagestyle{plain}

%%-----------------------------------------------------------------------
%%      INÍCIO DO DOCUMENTO
%%-----------------------------------------------------------------------

\setlength{\absparsep}{18pt}